\documentclass[12pt]{scrartcl}

%\usepackage[T1]{fontenc}
%\usepackage{mathpazo}
\usepackage{amsmath,amsfonts}

\usepackage{fontspec,xltxtra,xunicode}
\defaultfontfeatures{Mapping=tex-text}
\setromanfont[Mapping=tex-text]{Hoefler Text}
\setsansfont[Scale=MatchLowercase,Mapping=tex-text]{Gill Sans MT}
\setmonofont[Scale=MatchLowercase]{Andale Mono}

%% Fixes for letterspacing that work with XeLaTeX
\newcommand{\allcapsspacing}[1]{{\addfontfeature{LetterSpace=7.5}#1}}
\newcommand{\smallcapsspacing}[1]{{\addfontfeature{LetterSpace=5.0}#1}}
\renewcommand{\textsc}[1]{\smallcapsspacing{\scshape{#1}}}

\usepackage{CV}
%\usepackage[letterspace=100]{microtype}

\newcommand{\tworowstwocolsitem}[4]{
\noindent
\begin{tabular*}{\textwidth}{@{\extracolsep{\fill}}lr}
		#1 & #2 \\
		#3 & #4 \\
\end{tabular*}\vspace{0.75\baselineskip}}

\newcommand{\educationitem}[4]{
\tworowstwocolsitem{#1}{#2}{\textit{#3}}{\textit{#4}}}

\newcommand{\teachingitem}[4]{
\tworowstwocolsitem{#1}{#2}{\textit{#3}}{#4}}

\newcommand{\researchitem}[4]{
\noindent
\begin{tabular*}{\textwidth}{@{\extracolsep{\fill}}lr}
		#1 & #2 \\
		\textit{#3}\\
\end{tabular*}\\Topic: #4\\\\}

\newcommand{\yearitem}[2]{
\noindent
\begin{tabular*}{\textwidth}{@{\extracolsep{\fill}}lr}
		#1 & #2 \\
\end{tabular*}\vspace{0.75\baselineskip}}

\usepackage{titlesec}
\titleformat{\section}{}{}{}{\hrule\vspace{0.75\baselineskip}\sffamily\Large\uppercase}
\titlespacing{\section}{0pt}{-5pt}{*1}

% enables reverse numbering
\usepackage[leftmargin=12pt, itemsep=0.75\baselineskip]{etaremune}

\begin{document}

\begin{center}
\Large{\sffamily\allcapsspacing{\uppercase{Brian D. Weitzner}}}

\normalsize{\textsc{Department of Chemical \& Biomolecular Engineering \\
Johns Hopkins University \\
3400 N. Charles Street, Baltimore, Maryland 21218 \\
\vspace{0.25\baselineskip}
brian.weitzner@jhu.edu}}
\end{center}

\section*{\allcapsspacing{Education}}
\educationitem{The Johns Hopkins University}{Baltimore, MD}{Ph.D. Chemical \& Biomolecular Engineering}{2015 (Expected)}
\educationitem{Cornell University}{Ithaca, NY}{B.S. Chemical \& Biomolecular Engineering, Minor Biomedical Engineering}{2009}

\section*{\allcapsspacing{Research Experience}}
\researchitem{Graduate Research Assistant, The Johns Hopkins University}{2009--2014}{Advisor: Dr. Jeffrey J. Gray}{Computational modeling of antibodies and antibody--antigen complexes}
\researchitem{Undergraduate Research Assistant, Cornell University}{2006--2009}{Advisors: Dr. Matthew P. DeLisa \& Dr. Jeffrey D. Varner}{Computational retargeting of an E3 ubiquitin ligase}
\researchitem{Undergraduate Research Assistant, Fox Chase Cancer Center}{2005--2009}{Advisor: Dr. Roland L. Dunbrack, Jr.}{Dimerization motifs of cytosolic sulfotransferases}
\researchitem{Howard Hughes Student Scientist, Fox Chase Cancer Center}{2004--2005}{Advisor: Dr. Roland L. Dunbrack, Jr.}{Agreement among automated quaternary structure prediction methods}

%\pagebreak
\section*{\allcapsspacing{Publications}}
\begin{etaremune}
\item \textbf{Weitzner BD}, Dunbrack RL, Jr, Gray JJ (2014) ``The origin of CDR H3 structural diversity,'' \textit{under review}

\item \textbf{Weitzner BD}, Kuroda D, Marze N, Xu J, Gray JJ (2014) ``Blind prediction performance of RosettaAntibody 3.0: Grafting, relaxation, kinematic loop modeling, and full CDR optimization,'' \textit{Proteins} 82(8), 1611-23.

\item Lyskov S, Chou F-C, Conch{\'u}ir S{\'O}, Der BS, Drew K, Kuroda D, Xu J, \textbf{Weitzner BD}, Renfrew PD, Sripakdeevong P, Borgo B, Havranek JJ, Kuhlman B, Kortemme T, Bonneau R, Gray JJ, Das R (2013) ``Serverification of Molecular Modeling Applications: The Rosetta Online Server That Includes Everyone (ROSIE),'' \textit{PLOS ONE} 8(5): e63906.

\item Baugh EH, Lyskov S, \textbf{Weitzner BD}, Gray JJ (2011) ``Real-time PyMOL visualization for Rosetta and PyRosetta,'' \textit{PLOS ONE} 6(8): e21931.

\item Chaudhury S, Berrondo M, \textbf{Weitzner BD}, Muthu P, Bergman H, Gray JJ (2011) ``Benchmarking and analysis of protein docking performance in Rosetta v3.2,'' \textit{PLOS ONE} 6(8): e22477.

\item Bourne PE, Beran B, Bi C, Bluhm W, Dunbrack R, Prlic A, Quinn G, Rose P, Shah R, Tao W, \textbf{Weitzner B}, Yukich, B (2010) ``Will Widgets and Semantic Tagging Change Computational Biology?'' \textit{PLoS Comput. Biol.} 6(2): e1000673.

\item \textbf{Weitzner B}, Meehan T, Xu Q, Dunbrack R (2009) ``An unusually small dimer interface is observed in all available crystal structures of cytosolic sulfotransferases,'' \textit{Proteins.} 75(2), 1097-134.
\end{etaremune}

\section*{\allcapsspacing{Selected Honors and Awards}}
\yearitem{Rosetta Service Award: Instructor at inaugural Rosetta Boot Camp}{2013}
\yearitem{Rosetta Service Award: Leader of transition of Rosetta source\\code to a new version control system}{2013}
\yearitem{American Institute of Chemists Student Award}{2009}
\yearitem{1$^\text{st}$ place in national AIChE Car Competition;\\first team to ever perform perfectly}{2008}
\yearitem{Howard Hughes Medical Institute Student Scientist Program;\\Fox Chase Cancer Center}{2004--2005}
\yearitem{Eagle Scout}{2003}

\section*{\allcapsspacing{Invited Seminars and Talks}}
\begin{etaremune}
\item Weitzner BD, Gray JJ (2014) ``Next-generation Antibody Modeling'' \textit{Seminar, Center for Biomolecular Structure and Dynamics, University of Montana}, Missoula, MT.

\item Weitzner BD, Dunbrack RL, Gray JJ (2014) ``The origin of CDR H3 Structural Diversity'' \textit{Vortrag, Fakult\"{a}t f\"{u}r Chemie, Technische Universit\"{a}t M\"{u}nchen}, Munich, Germany.

\item Weitzner BD, Gray JJ (2013) ``Computational Structure Prediction, Docking and Design of Antibodies'' \textit{IBC Antibody Engineering and Therapeutics}, Huntington Beach, CA. (delivered on behalf of JJG during his paternity leave)
\end{etaremune}

\section*{\allcapsspacing{Scientific Meetings and Talks}}
\begin{etaremune}
\item Weitzner BD, Kuroda D, Marze N, Xu J, Gray JJ (2013) ``Benchmarking RosettaAntibody: Antibody Modeling Assessment II'' \textit{Antibody Engineering and Therapeutics}, Huntington Beach, CA.

\item Weitzner BD, Roland RL, Gray JJ (2013) ``Kinked CDR H3-like loops are common'' \textit{AIChE}, San Francisco, CA

\item Weitzner BD, Dunbrack RL, Gray JJ (2013) ``Antibodies are proteins too!'' \textit{Rosetta Conference}, Leavenworth, WA.

\item Lyskov S, Wetizner BD, Gray JJ (2011) ``PyRosetta 2.0: I can make a new score term in 6 lines!'' \textit{Rosetta Conference}, Leavenworth, WA.

\item Weitzner BD, Leaver-fay A, Kulp D,  Lyskov S (2010) ``Using PyRosetta for research'' \textit{Rosetta Conference}, Leavenworth, WA. [workshop]

\item Weitzner BD, Baugh EH, Gray JJ (2010) ``PyMOL--PyRosetta Integration'' \textit{Rosetta Conference}, Leavenworth, WA.
\end{etaremune}

%\pagebreak
\section*{\textsc{Scientific Posters}}
\begin{etaremune}
\item Weitzner, BD, Dunbrack RL, Gray JJ (2014) ``CDR H3 loop prediction'' \textit{Rosetta Conference}, Leavenworth, WA.

\item Weitzner, BD, Dunbrack RL, Gray JJ (2012) ``Are CDR H3 loops special?'' \textit{Rosetta Conference}, Leavenworth, WA.

\item Weitzner, BD, Dunbrack RL, Gray JJ (2011) ``Accessing the conformation space of long CDR H3 loops through $\beta$-turn detection'' \textit{Rosetta Conference}, Leavenworth, WA.
\end{etaremune}

\section*{\allcapsspacing{Teaching Experience}}
\teachingitem{Co-Instructor, Rosetta Boot Camp}{Spring 2013}{An intense week-long crash course to developing in Rosetta}{Chapel Hill, NC}
\teachingitem{Co-Instructor, ChemBE 418}{Fall 2011}{Projects in the Design of a Chemical Car}{Johns Hopkins University}
\teachingitem{Teaching Assistant, ChemBE 409}{Fall 2010}{Modeling, Dynamics \& Control of Chemical \& Biological Systems}{Johns Hopkins University}
\teachingitem{Teaching Assistant, ChemBE 414/614}{Fall 2010}{Computational Protein Structure Prediction and Design}{Johns Hopkins University}
\teachingitem{Teaching Assistant, ChemE 3900}{Spring 2009}{Chemical Kinetics and Reactor Design}{Cornell University}
\teachingitem{Teaching Assistant, ChemE 1120}{Fall 2008}{Introduction to Chemical Engineering}{Cornell University}

\section*{\allcapsspacing{Activities and Outreach}}
\begin{CV}
\item[2013--2014] Student volunteer, STEM Achievement in Baltimore Elementary Schools (SABES) Program
\item[2010--2011] Member, Rosetta XRW Team to overhaul the structure of the source code
\item[Fall 2010] Ricky Myers Day of Service
\item[2009--2013] Member, JHU ChemBE department STEM outreach group
\item[2008--2009] Captain, AIChE Car Team, Cornell University
\item[2006--2008] Member, AIChE Car Team, Cornell University
\end{CV}
\end{document}
